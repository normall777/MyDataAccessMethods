\documentclass[]{article}
\usepackage{lmodern}
\usepackage{amssymb,amsmath}
\usepackage{ifxetex,ifluatex}
\usepackage{fixltx2e} % provides \textsubscript
\usepackage[T2A]{fontenc}
\usepackage[english, russian]{babel}
\ifnum 0\ifxetex 1\fi\ifluatex 1\fi=0 % if pdftex
  \usepackage[T1]{fontenc}
  \usepackage[utf8]{inputenc}
\else % if luatex or xelatex
  \ifxetex
    \usepackage{mathspec}
  \else
    \usepackage{fontspec}
  \fi
  \defaultfontfeatures{Ligatures=TeX,Scale=MatchLowercase}
\fi
% use upquote if available, for straight quotes in verbatim environments
\IfFileExists{upquote.sty}{\usepackage{upquote}}{}
% use microtype if available
\IfFileExists{microtype.sty}{%
\usepackage{microtype}
\UseMicrotypeSet[protrusion]{basicmath} % disable protrusion for tt fonts
}{}
\usepackage[margin=1in]{geometry}
\usepackage{hyperref}
\hypersetup{unicode=true,
            pdftitle={Домашняя работа 1},
            pdfauthor={Izotov Ilya},
            pdfborder={0 0 0},
            breaklinks=true}
\urlstyle{same}  % don't use monospace font for urls
\usepackage{color}
\usepackage{fancyvrb}
\newcommand{\VerbBar}{|}
\newcommand{\VERB}{\Verb[commandchars=\\\{\}]}
\DefineVerbatimEnvironment{Highlighting}{Verbatim}{commandchars=\\\{\}}
% Add ',fontsize=\small' for more characters per line
\usepackage{framed}
\definecolor{shadecolor}{RGB}{248,248,248}
\newenvironment{Shaded}{\begin{snugshade}}{\end{snugshade}}
\newcommand{\KeywordTok}[1]{\textcolor[rgb]{0.13,0.29,0.53}{\textbf{#1}}}
\newcommand{\DataTypeTok}[1]{\textcolor[rgb]{0.13,0.29,0.53}{#1}}
\newcommand{\DecValTok}[1]{\textcolor[rgb]{0.00,0.00,0.81}{#1}}
\newcommand{\BaseNTok}[1]{\textcolor[rgb]{0.00,0.00,0.81}{#1}}
\newcommand{\FloatTok}[1]{\textcolor[rgb]{0.00,0.00,0.81}{#1}}
\newcommand{\ConstantTok}[1]{\textcolor[rgb]{0.00,0.00,0.00}{#1}}
\newcommand{\CharTok}[1]{\textcolor[rgb]{0.31,0.60,0.02}{#1}}
\newcommand{\SpecialCharTok}[1]{\textcolor[rgb]{0.00,0.00,0.00}{#1}}
\newcommand{\StringTok}[1]{\textcolor[rgb]{0.31,0.60,0.02}{#1}}
\newcommand{\VerbatimStringTok}[1]{\textcolor[rgb]{0.31,0.60,0.02}{#1}}
\newcommand{\SpecialStringTok}[1]{\textcolor[rgb]{0.31,0.60,0.02}{#1}}
\newcommand{\ImportTok}[1]{#1}
\newcommand{\CommentTok}[1]{\textcolor[rgb]{0.56,0.35,0.01}{\textit{#1}}}
\newcommand{\DocumentationTok}[1]{\textcolor[rgb]{0.56,0.35,0.01}{\textbf{\textit{#1}}}}
\newcommand{\AnnotationTok}[1]{\textcolor[rgb]{0.56,0.35,0.01}{\textbf{\textit{#1}}}}
\newcommand{\CommentVarTok}[1]{\textcolor[rgb]{0.56,0.35,0.01}{\textbf{\textit{#1}}}}
\newcommand{\OtherTok}[1]{\textcolor[rgb]{0.56,0.35,0.01}{#1}}
\newcommand{\FunctionTok}[1]{\textcolor[rgb]{0.00,0.00,0.00}{#1}}
\newcommand{\VariableTok}[1]{\textcolor[rgb]{0.00,0.00,0.00}{#1}}
\newcommand{\ControlFlowTok}[1]{\textcolor[rgb]{0.13,0.29,0.53}{\textbf{#1}}}
\newcommand{\OperatorTok}[1]{\textcolor[rgb]{0.81,0.36,0.00}{\textbf{#1}}}
\newcommand{\BuiltInTok}[1]{#1}
\newcommand{\ExtensionTok}[1]{#1}
\newcommand{\PreprocessorTok}[1]{\textcolor[rgb]{0.56,0.35,0.01}{\textit{#1}}}
\newcommand{\AttributeTok}[1]{\textcolor[rgb]{0.77,0.63,0.00}{#1}}
\newcommand{\RegionMarkerTok}[1]{#1}
\newcommand{\InformationTok}[1]{\textcolor[rgb]{0.56,0.35,0.01}{\textbf{\textit{#1}}}}
\newcommand{\WarningTok}[1]{\textcolor[rgb]{0.56,0.35,0.01}{\textbf{\textit{#1}}}}
\newcommand{\AlertTok}[1]{\textcolor[rgb]{0.94,0.16,0.16}{#1}}
\newcommand{\ErrorTok}[1]{\textcolor[rgb]{0.64,0.00,0.00}{\textbf{#1}}}
\newcommand{\NormalTok}[1]{#1}
\usepackage{graphicx,grffile}
\makeatletter
\def\maxwidth{\ifdim\Gin@nat@width>\linewidth\linewidth\else\Gin@nat@width\fi}
\def\maxheight{\ifdim\Gin@nat@height>\textheight\textheight\else\Gin@nat@height\fi}
\makeatother
% Scale images if necessary, so that they will not overflow the page
% margins by default, and it is still possible to overwrite the defaults
% using explicit options in \includegraphics[width, height, ...]{}
\setkeys{Gin}{width=\maxwidth,height=\maxheight,keepaspectratio}
\IfFileExists{parskip.sty}{%
\usepackage{parskip}
}{% else
\setlength{\parindent}{0pt}
\setlength{\parskip}{6pt plus 2pt minus 1pt}
}
\setlength{\emergencystretch}{3em}  % prevent overfull lines
\providecommand{\tightlist}{%
  \setlength{\itemsep}{0pt}\setlength{\parskip}{0pt}}
\setcounter{secnumdepth}{0}
% Redefines (sub)paragraphs to behave more like sections
\ifx\paragraph\undefined\else
\let\oldparagraph\paragraph
\renewcommand{\paragraph}[1]{\oldparagraph{#1}\mbox{}}
\fi
\ifx\subparagraph\undefined\else
\let\oldsubparagraph\subparagraph
\renewcommand{\subparagraph}[1]{\oldsubparagraph{#1}\mbox{}}
\fi

%%% Use protect on footnotes to avoid problems with footnotes in titles
\let\rmarkdownfootnote\footnote%
\def\footnote{\protect\rmarkdownfootnote}

%%% Change title format to be more compact
\usepackage{titling}

% Create subtitle command for use in maketitle
\newcommand{\subtitle}[1]{
  \posttitle{
    \begin{center}\large#1\end{center}
    }
}

\setlength{\droptitle}{-2em}

  \title{"Домашняя работа 1"}
    \pretitle{\vspace{\droptitle}\centering\huge}
  \posttitle{\par}
    \author{Izotov Ilya}
    \preauthor{\centering\large\emph}
  \postauthor{\par}
      \predate{\centering\large\emph}
  \postdate{\par}
    \date{"26 сентября 2018 г"}


\begin{document}
\maketitle

\subsection{Работа с данными.}\label{--.}

По адресу
\url{http://people.math.umass.edu/~anna/Stat597AFall2016/rnf6080.dat}
можно получить набор данных об осадках в Канаде с 1960 по 1980 годы.
Необходимо загрузить эти данные при помощи \texttt{read.table}.
Воспользуйтесь справкой, чтобы изучить аргументы, которые принимает
функция.

\begin{itemize}
\tightlist
\item
  Загрузите данные в датафрейм, который назовите \texttt{data.df}.
\end{itemize}

\begin{Shaded}
\begin{Highlighting}[]
\NormalTok{data.df <-}\StringTok{ }\KeywordTok{read.table}\NormalTok{(}\StringTok{"http://people.math.umass.edu/~anna/Stat597AFall2016/rnf6080.dat"}\NormalTok{)}
\end{Highlighting}
\end{Shaded}

\begin{itemize}
\tightlist
\item
  Сколько строк и столбцов в \texttt{data.df}? Если получилось не 5070
  наблюдений 27 переменных, то проверяйте аргументы.
\end{itemize}

\begin{Shaded}
\begin{Highlighting}[]
\KeywordTok{length}\NormalTok{(data.df[}\DecValTok{1}\NormalTok{,]); }\KeywordTok{length}\NormalTok{(data.df[,}\DecValTok{1}\NormalTok{])}
\end{Highlighting}
\end{Shaded}

\begin{verbatim}
## [1] 27
\end{verbatim}

\begin{verbatim}
## [1] 5070
\end{verbatim}

\begin{itemize}
\tightlist
\item
  Получите имена колонок из \texttt{data.df}.
\end{itemize}

\begin{Shaded}
\begin{Highlighting}[]
\KeywordTok{names}\NormalTok{(data.df)}
\end{Highlighting}
\end{Shaded}

\begin{verbatim}
##  [1] "V1"  "V2"  "V3"  "V4"  "V5"  "V6"  "V7"  "V8"  "V9"  "V10" "V11"
## [12] "V12" "V13" "V14" "V15" "V16" "V17" "V18" "V19" "V20" "V21" "V22"
## [23] "V23" "V24" "V25" "V26" "V27"
\end{verbatim}

\begin{verbatim}
Как видно, стандартные имена столбцов записываются как V1, V2...
\end{verbatim}

\begin{itemize}
\tightlist
\item
  Найдите значение из 5 строки седьмого столбца.
\end{itemize}

\begin{Shaded}
\begin{Highlighting}[]
\KeywordTok{paste}\NormalTok{(data.df[}\DecValTok{5}\NormalTok{,}\DecValTok{7}\NormalTok{])}
\end{Highlighting}
\end{Shaded}

\begin{verbatim}
## [1] "0"
\end{verbatim}

\begin{itemize}
\tightlist
\item
  Напечатайте целиком 2 строку из \texttt{data.df}
\end{itemize}

\begin{Shaded}
\begin{Highlighting}[]
\KeywordTok{paste}\NormalTok{(data.df[}\DecValTok{2}\NormalTok{,])}
\end{Highlighting}
\end{Shaded}

\begin{verbatim}
##  [1] "60" "4"  "2"  "0"  "0"  "0"  "0"  "0"  "0"  "0"  "0"  "0"  "0"  "0" 
## [15] "0"  "0"  "0"  "0"  "0"  "0"  "0"  "0"  "0"  "0"  "0"  "0"  "0"
\end{verbatim}

\begin{itemize}
\tightlist
\item
  Объясните, что делает следующая строка кода
  \texttt{names(data.df)\ \textless{}-\ c("year",\ "month",\ "day",\ seq(0,23))}.
  Воспользуйтесь функциями \texttt{head} и \texttt{tail}, чтобы
  просмотреть таблицу. Что представляют собой последние 24 колонки?
\end{itemize}

\begin{Shaded}
\begin{Highlighting}[]
\KeywordTok{names}\NormalTok{(data.df) <-}\StringTok{ }\KeywordTok{c}\NormalTok{(}\StringTok{"year"}\NormalTok{, }\StringTok{"month"}\NormalTok{, }\StringTok{"day"}\NormalTok{, }\KeywordTok{seq}\NormalTok{(}\DecValTok{0}\NormalTok{,}\DecValTok{23}\NormalTok{))}
\KeywordTok{head}\NormalTok{(data.df)}
\end{Highlighting}
\end{Shaded}

\begin{verbatim}
##   year month day 0 1 2 3 4 5 6 7 8 9 10 11 12 13 14 15 16 17 18 19 20 21
## 1   60     4   1 0 0 0 0 0 0 0 0 0 0  0  0  0  0  0  0  0  0  0  0  0  0
## 2   60     4   2 0 0 0 0 0 0 0 0 0 0  0  0  0  0  0  0  0  0  0  0  0  0
## 3   60     4   3 0 0 0 0 0 0 0 0 0 0  0  0  0  0  0  0  0  0  0  0  0  0
## 4   60     4   4 0 0 0 0 0 0 0 0 0 0  0  0  0  0  0  0  0  0  0  0  0  0
## 5   60     4   5 0 0 0 0 0 0 0 0 0 0  0  0  0  0  0  0  0  0  0  0  0  0
## 6   60     4   6 0 0 0 0 0 0 0 0 0 0  0  0  0  0  0  0  0  0  0  0  0  0
##   22 23
## 1  0  0
## 2  0  0
## 3  0  0
## 4  0  0
## 5  0  0
## 6  0  0
\end{verbatim}

\begin{Shaded}
\begin{Highlighting}[]
\KeywordTok{tail}\NormalTok{(data.df)}
\end{Highlighting}
\end{Shaded}

\begin{verbatim}
##      year month day 0 1 2 3 4 5 6 7 8 9 10 11 12 13 14 15 16 17 18 19 20
## 5065   80    11  25 0 0 0 0 0 0 0 0 0 0  0  0  0  0  0  0  0  0  0  0  0
## 5066   80    11  26 0 0 0 0 0 0 0 0 0 0  0  0  0  0  0  0  0  0  0  0  0
## 5067   80    11  27 0 0 0 0 0 0 0 0 0 0  0  0  0  0  0  0  0  0  0  0  0
## 5068   80    11  28 0 0 0 0 0 0 0 0 0 0  0  0  0  0  0  0  0  0  0  0  0
## 5069   80    11  29 0 0 0 0 0 0 0 0 0 0  0  0  0  0  0  0  0  0  0  0  0
## 5070   80    11  30 0 0 0 0 0 0 0 0 0 0  0  0  0  0  0  0  0  0  0  0  0
##      21 22 23
## 5065  0  0  0
## 5066  0  0  0
## 5067  0  0  0
## 5068  0  0  0
## 5069  0  0  0
## 5070  0  0  0
\end{verbatim}

\begin{verbatim}
Первая строка дает колонкам названия. 
1 колонка - год, 2 - месяц, 3 - день, остальные - час от 0 до 23. 
Head показывает верхнюю часть таблицы, а tail - нижнюю. 
\end{verbatim}

\begin{itemize}
\tightlist
\item
  Добавьте новую колонку с названием \emph{daily}, в которую запишите
  сумму крайних правых 24 колонок. Постройте гистограмму по этой
  колонке. Какие выводы можно сделать?
\end{itemize}

\begin{Shaded}
\begin{Highlighting}[]
\NormalTok{data.df.daily <-}\StringTok{ }\KeywordTok{data.frame}\NormalTok{(data.df, }\DataTypeTok{daily=}\KeywordTok{rowSums}\NormalTok{(data.df)}\OperatorTok{-}\NormalTok{data.df}\OperatorTok{$}\NormalTok{year}\OperatorTok{-}\NormalTok{data.df}\OperatorTok{$}\NormalTok{month}\OperatorTok{-}\NormalTok{data.df}\OperatorTok{$}\NormalTok{day)}
\KeywordTok{hist}\NormalTok{(data.df.daily}\OperatorTok{$}\NormalTok{daily)}
\end{Highlighting}
\end{Shaded}

\includegraphics{hw01_files/figure-latex/unnamed-chunk-7-1.pdf}

\begin{verbatim}
В данных имеются отрицательные значения. Эти данные необходимо обнулить, т.к. таких значений быть на практике не может.
\end{verbatim}

\begin{itemize}
\tightlist
\item
  Создайте новый датафрейм \texttt{fixed.df} в котром исправьте
  замеченную ошибку. Постройте новую гистограмму, поясните почему она
  более корректна.
\end{itemize}

\begin{Shaded}
\begin{Highlighting}[]
\NormalTok{fixed.df <-}\StringTok{ }\NormalTok{data.df.daily}
\NormalTok{fixed.df[fixed.df }\OperatorTok{<}\StringTok{ }\DecValTok{0}\NormalTok{] <-}\StringTok{ }\DecValTok{0}
\KeywordTok{hist}\NormalTok{(fixed.df}\OperatorTok{$}\NormalTok{daily)}
\end{Highlighting}
\end{Shaded}

\includegraphics{hw01_files/figure-latex/unnamed-chunk-8-1.pdf}

\begin{verbatim}
Получившаяся гистограмма более корректна, поскольку имеет только неотрицательные значения. Ошибочные отрицательные значения приравнены к 0.
\end{verbatim}

\subsection{Синтаксис и типизирование}\label{--}

\begin{itemize}
\tightlist
\item
  Для каждой строки кода поясните полученный результат, либо объясните
  почему она ошибочна.
\end{itemize}

\begin{verbatim}
v <- c("4", "8", "15", "16", "23", "42")
max(v)
sort(v)
sum(v)
\end{verbatim}

\texttt{v\ \textless{}-\ c("4",\ "8",\ "15",\ "16",\ "23",\ "42")} - в
переменную v помещается вектор символов (char)

\texttt{max(v)} - поиск наибольшего символа, с которого начнется строка.
В кодировке код цифры 8 стоит после цифр 0-7, поэтому максимальным
элементом вектора будет выбран символ ``8''

\texttt{sort(v)} - сортировка элементов вектора по возрастанию. Символ
``8'' будет последним, поскольку он максимальный. ``15'' - будет первым,
``16'' - вторым, так как оба элемента начинаются на одинаковый символ,
то будет сравнение по второму символу. И так дальше.

\texttt{sum(v)} - выполнено не будет, так как тип вектора символьный, а
не числовой.

\begin{itemize}
\tightlist
\item
  Для следующих наборов команд поясните полученный результат, либо
  объясните почему они ошибочна.
\end{itemize}

\begin{verbatim}
#Набор команд 1
v2 <- c("5",7,12)
v2[2] + v2[3]

#Набор команд 2
df3 <- data.frame(z1="5",z2=7,z3=12)
df3[1,2] + df3[1,3]

#Набор команд 3
l4 <- list(z1="6", z2=42, z3="49", z4=126)
l4[[2]] + l4[[4]]
l4[2] + l4[4]
\end{verbatim}

Набор команд 1 - вторая команда не будет выполнена, поскольку при
инициализации переменной v2 в нее был передан элемент типа char. Вектор
может хранить в себе элементы только одного типа, поэтому остальные
элементы вектора тоже стали типа char. Суммировать элементы типа char
оператором + не получится.

Набор команд 2 - создает датафрейм размером 1х3. Датафрейм может иметь в
своем составе разные типы элементов, поэтому в данном случае будет
успешно выполнено сложение двух чисел.

Набор команд 3 - создает список элементов. Элементы так же могут быть
разных типов. Первое сложение успешно выполняется, поскольку указывается
конкретный порядковый номер элемента списка. Следующая команда выполнена
не будет, поскольку неправильное обращение к элементу. Правильным
вариантом будет \texttt{l4\$z2{[}1{]}+l4\$z4{[}1{]}}.

\subsection{Работа с функциями и операторами}\label{----}

\begin{itemize}
\tightlist
\item
  Оператор двоеточие создаёт последовательность целых чисел по порядку.
  Этот оператор --- частный случай функции \texttt{seq()}, которую вы
  использовали раньше. Изучите эту функцию, вызвав команду
  \texttt{?seq}. Испольуя полученные знания выведите на экран:
\end{itemize}

\begin{enumerate}
\def\labelenumi{\arabic{enumi}.}
\tightlist
\item
  Числа от 1 до 10000 с инкрементом 372.
\end{enumerate}

\begin{Shaded}
\begin{Highlighting}[]
\KeywordTok{seq}\NormalTok{(}\DataTypeTok{from =} \DecValTok{1}\NormalTok{, }\DataTypeTok{to =} \DecValTok{10000}\NormalTok{, }\DataTypeTok{by=}\DecValTok{372}\NormalTok{)}
\end{Highlighting}
\end{Shaded}

\begin{verbatim}
##  [1]    1  373  745 1117 1489 1861 2233 2605 2977 3349 3721 4093 4465 4837
## [15] 5209 5581 5953 6325 6697 7069 7441 7813 8185 8557 8929 9301 9673
\end{verbatim}

\begin{enumerate}
\def\labelenumi{\arabic{enumi}.}
\setcounter{enumi}{1}
\tightlist
\item
  Числа от 1 до 10000 длиной 50.
\end{enumerate}

\begin{Shaded}
\begin{Highlighting}[]
\KeywordTok{seq}\NormalTok{(}\DataTypeTok{from=}\DecValTok{1}\NormalTok{, }\DataTypeTok{to=}\DecValTok{10000}\NormalTok{, }\DataTypeTok{length.out =} \DecValTok{50}\NormalTok{)}
\end{Highlighting}
\end{Shaded}

\begin{verbatim}
##  [1]     1.0000   205.0612   409.1224   613.1837   817.2449  1021.3061
##  [7]  1225.3673  1429.4286  1633.4898  1837.5510  2041.6122  2245.6735
## [13]  2449.7347  2653.7959  2857.8571  3061.9184  3265.9796  3470.0408
## [19]  3674.1020  3878.1633  4082.2245  4286.2857  4490.3469  4694.4082
## [25]  4898.4694  5102.5306  5306.5918  5510.6531  5714.7143  5918.7755
## [31]  6122.8367  6326.8980  6530.9592  6735.0204  6939.0816  7143.1429
## [37]  7347.2041  7551.2653  7755.3265  7959.3878  8163.4490  8367.5102
## [43]  8571.5714  8775.6327  8979.6939  9183.7551  9387.8163  9591.8776
## [49]  9795.9388 10000.0000
\end{verbatim}

\begin{itemize}
\tightlist
\item
  Функция \texttt{rep()} повторяет переданный вектор указанное число
  раз. Объясните разницу между \texttt{rep(1:5,times=3)} и
  \texttt{rep(1:5,\ each=3)}.
\end{itemize}

\begin{Shaded}
\begin{Highlighting}[]
\KeywordTok{rep}\NormalTok{(}\DecValTok{1}\OperatorTok{:}\DecValTok{5}\NormalTok{, }\DataTypeTok{times=}\DecValTok{3}\NormalTok{)}
\end{Highlighting}
\end{Shaded}

\begin{verbatim}
##  [1] 1 2 3 4 5 1 2 3 4 5 1 2 3 4 5
\end{verbatim}

Данный вектор будет повторяться в том порядке, в котором передан был
изначально. В times указано количество повторений.

\begin{Shaded}
\begin{Highlighting}[]
\KeywordTok{rep}\NormalTok{(}\DecValTok{1}\OperatorTok{:}\DecValTok{5}\NormalTok{, }\DataTypeTok{each=}\DecValTok{3}\NormalTok{)}
\end{Highlighting}
\end{Shaded}

\begin{verbatim}
##  [1] 1 1 1 2 2 2 3 3 3 4 4 4 5 5 5
\end{verbatim}

В данном случае каждый элемент вектора будет повторен сразу друг за
другом в количестве раз, указанном в each.


\end{document}
